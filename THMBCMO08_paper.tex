% !TeX spellcheck = en_US
%
%
% v 2.3  Feb 2019   Volker RW Schaa
%		# changes in the collaboration therefore updated file "jacow-collaboration.tex"
%		# all References with DOIs have their period/full stop before the DOI (after pp. or year)
%		# in the author/affiliation block all ZIP codes in square brackets removed as it was not 
%         understood as optional parameter and ZIP codes had bin put in brackets
%       # References to the current IPAC are changed to "IPAC'19, Melbourne, Australia"
%       # font for "url" style changed to "newtxtt" as it is easier to distinguish "O" and "0"
%
\documentclass[letter,
               %boxit,        % check whether paper is inside correct margins
               %titlepage,    % separate title page
               %refpage       % separate references
               %biblatex,     % biblatex is used
               keeplastbox,   % flushend option: not to un-indent last line in References
               %nospread,     % flushend option: do not fill with whitespace to balance columns
               %hyphens,      % allow \url to hyphenate at "-" (hyphens)
               %xetex,        % use XeLaTeX to process the file
               %luatex,       % use LuaLaTeX to process the file
               ]{jacow}
%
% ONLY FOR \footnote in table/tabular
%
\usepackage{pdfpages,multirow,ragged2e} %
%
% CHANGE SEQUENCE OF GRAPHICS EXTENSION TO BE EMBEDDED
% ----------------------------------------------------
% test for XeTeX where the sequence is by default eps-> pdf, jpg, png, pdf, ...
%    and the JACoW template provides JACpic2v3.eps and JACpic2v3.jpg which
%    might generates errors, therefore PNG and JPG first
%
\makeatletter%
	\ifboolexpr{bool{xetex}}
	 {\renewcommand{\Gin@extensions}{.pdf,%
	                    .png,.jpg,.bmp,.pict,.tif,.psd,.mac,.sga,.tga,.gif,%
	                    .eps,.ps,%
	                    }}{}
\makeatother

% CHECK FOR XeTeX/LuaTeX BEFORE DEFINING AN INPUT ENCODING
% --------------------------------------------------------
%   utf8  is default for XeTeX/LuaTeX
%   utf8  in LaTeX only realises a small portion of codes
%
\ifboolexpr{bool{xetex} or bool{luatex}} % test for XeTeX/LuaTeX
 {}                                      % input encoding is utf8 by default
 {\usepackage[utf8]{inputenc}}           % switch to utf8

\usepackage[USenglish]{babel}

% My packages:
\usepackage{listings}
\usepackage{color}

\definecolor{dkgreen}{rgb}{0,0.6,0}
\definecolor{gray}{rgb}{0.5,0.5,0.5}
\definecolor{mauve}{rgb}{0.58,0,0.82}

\lstset{frame=tb,
  language=Python,
  aboveskip=3mm,
  belowskip=3mm,
  showstringspaces=false,
  columns=flexible,
  basicstyle={\small\ttfamily},
  numbers=none,
  numberstyle=\tiny\color{gray},
  keywordstyle=\color{blue},
  commentstyle=\color{mauve},
  stringstyle=\color{dkgreen},
  breaklines=true,
  breakatwhitespace=true,
  tabsize=3
}

%
% if BibLaTeX is used
%
\ifboolexpr{bool{jacowbiblatex}}%
 {%
  \addbibresource{jacow-test.bib}
  \addbibresource{biblatex-examples.bib}
 }{}
\listfiles

%%
%%   Lengths for the spaces in the title
%%   \setlength\titleblockstartskip{..}  %before title, default 3pt
%%   \setlength\titleblockmiddleskip{..} %between title + author, default 1em
%%   \setlength\titleblockendskip{..}    %afterauthor, default 1em

\begin{document}

\title{WHATRECORD: A PYTHON-BASED EPICS FILE FORMAT TOOL
\thanks{Work supported by U.S. D.O.E. Contract DE-AC02-76SF00515.}}
\author{Kenneth Lauer\thanks{klauer@slac.stanford.edu}, SLAC National Accelerator Laboratory, Menlo Park, CA }
	
\maketitle

%
\begin{abstract}
  whatrecord is a Python-based parsing tool for interacting with a variety of
  EPICS (Experimental Physics and Industrial Control System) file formats,
  including V3 and V7 database files. The project aims for compliance with
  epics-base by using Lark\cite{lark} grammars that closely reflect the
  original Lex/Yacc grammars. 

  whatrecord offers a suite of tools for working with its supported file
  formats, with convenient Python-facing dataclass object representations and
  easy JSON (JavaScript Object Notation) serialization. A prototype backend web
  server for hosting IOC (Input/Output Controller) and record information is
  also included as well as a Vue.js-based frontend, an EPICS build system
  Makefile dependency inspector, a static analyzer-of-sorts for startup
  scripts, and a host of other things that the author added at whim to this
  side project.
\end{abstract}


\section{BACKGROUND}

\begin{figure}[!htb]
   \centering
   \includegraphics*[width=.7\columnwidth]{JACpic_mc}
   \caption{Layout of papers.}
   \label{fig:paper_layout}
\end{figure}

\subsection{The problem - and the inspiration}

At the LCLS, the accelerator and photon side control systems include
approximately 3000 IOC instances in total, with hundreds of modules and
dozens of versions per module.

In general, these EPICS IOCs, modules, and extensions are comprised of a
conglomeration of unique file formats. Some examples include:

\begin{Itemize}
  \item Process database files (.db)
  \item Database definition files (.dbd)
  \item Template / substitutions files
  \item IOC shell scripts (st.cmd)
  \item StreamDevice protocols (.proto)
  \item State notation language programs (.st)
  \item Gateway configuration (.pvlist)
  \item Access security files (.acf)
  \item Facility-specific things like LCLS’s IOC manager configuration
  \item Build system Makefiles
\end{Itemize}

Combined, this makes for an enormous code base with a mix of these
EPICS-specific file formats.

Links between these files are often implicit.  Take, for example, that an EPICS
IOC record has a specific record type name alongside its name in a database
file (.db).  An EPICS PV (Process Variable) name, in a traditional IOC, starts
with the record name defined in a database file. This PV name acts as a global
identifier that allows for clients on the same network subnet to access - and
potentially modify - related data.

A record is made up of fields which can contain metadata like engineering units
or user-specified descriptions, references to other records, relevant data
values, and so on.

An example record instance, defining a single AI (analog input) record named
\verb_IOC:RECORD:NAME_ is as follows:
\begin{lstlisting}[language=bash]
  record(ai, "IOC:RECORD:NAME") {}
\end{lstlisting}

This file does not define what the fields of the record type; that is the
responsibility of the database definition file (.dbd). A simplified excerpt
from a database definition file, defining a single field
for the "ai" record type is as follows:

\begin{lstlisting}[language=bash]
  recordtype(ai) {
      ...
      field(NAME, DBF_STRING) {
          special(SPC_NOMOD)
          size(61)
          prompt("Record Name")
      }
      ...
  }
\end{lstlisting}

Note that there is no explicit link between the database file and the database
definition file: neither reference the other by filename.
Rather, one can only infer the link by examining a third file, the IOC-specific
IOC shell script (.cmd) file, line-by-line.

An excerpt from such a startup script could look like:
\begin{lstlisting}[language=bash]
dbLoadDatabase("path/to/the.dbd",0,0)
IOC_registerRecordDeviceDriver(pdbbase) 
dbLoadRecords("records.db")
\end{lstlisting}

Each line of this script includes up to one command. Each of those commands
has been registered by either EPICS itself, the modules included in the IOC,
or the IOC source code itself.  Typically, the available commands
would be found either in documentation or by executing the IOC and invoking the
built-in help system. Alternatively, the most reliable fallback ends up
being the source code itself.

Other direct or indirect references may be found inside fields.  
For example, depending on the \verb_DTYP_ (device type) field, the \verb_INP_
(input specification) field may be a custom string defined at the device
support layer. Interpretation of this field requires knowledge of how
these are formatted.  Take StreamDevice\cite{streamdevice}, a generic
support module for communicating with controllers that use simple byte
streams for communication, for example:
\begin{lstlisting}[language=bash]
  record(ai, "IOC:RECORD:NAME") {
    field(DTYP, "stream")
    field(INP,  "@ProtocolFilename.proto getValue PS1")
  }
\end{lstlisting}
The device type here is set to \verb_"stream"_, a custom identifier that
StreamDevice has hard-coded.  This instructs EPICS to use StreamDevice
and interpret the \verb_INP_ field with it.  It us up to the IOC developer to
understand the format of these strings and set them appropriately, in order
to reference back to the protocol file that defines the byte string to send
and the expected response format. Here, a StreamDevice protocol file for
the above record indicates that a simple string \verb_WHAT:IS:THE:VALUE?_
is to be sent, and a floating point value (\verb_%f_, as in the C scanf
format specifiers) is to be sent from the controller:
\begin{lstlisting}[language=bash]
  getValue {
      out "WHAT:IS:THE:VALUE?"; in "%f";
  }
\end{lstlisting}

This section is a small but important part of what makes up an IOC: the build
system surrounding all of these files, other modules with their own standards,
access security configuration for intra-subnet access, gateway configuration
controlling inter-subnet access, facility-specific tools that rely on PV
names, and so on further complicate the number of files and references
one needs to be aware of.

While those familiar with EPICS IOC development may find that the above is
obvious and simple, it can be opaque at best to newer users and those unable to
dedicate the time to reading through esoteric manuals (if lucky, source code
otherwise).

\subsection{Goals and the emergence of whatrecord}

The previous section's problem led the author over the years to desire a tool
that could somehow unify these file formats and provide the ability to inspect
the implicit links.

The overall goals of the project could be summarized as:
\begin{itemize}
  \item Allow for easy parsing of all the special formats outside, and
    represent them in a widely-used interchange format like JSON.
  \item Aid the user in the understanding of existing IOCs, whether they are
    deployed and running or not.
  \item Provide a method to see how different records, different IOCs, all
    relate to one another, without requiring the IOC to be running.
  \item Provide a method for cross-referencing a PV name to its database
    file, record definition, startup script, and IOC.
\end{itemize}

These initial goals led to the creation of this new Python package, whatrecord.
Taking it a step further and exploring possibilities led to linking records to
PLC code, to StreamDevice protocol information, to gateway access rules, 
and even shell commands to their respective source code.

\subsection{Features}
whatrecord will parse any of the following into intuitive Python dataclasses using lark:
\begin{itemize}
  \item Database files (V3 or V4/V7), database definitions, template/substitution files
  \item Access security configuration files
  \item Autosave .sav files
  \item Gateway pvlist configuration files
  \item StreamDevice protocol files
  \item snlseq/sequencer state machine parsing
\end{itemize}

whatrecord can also interpret IOC shell scripts (i.e., st.cmd) and track
contextual information during the loading process:
\begin{itemize}
  \item What files were loaded during startup?
  \item What records are available?
  \item What errors were found?
  \item What file and line did record X get loaded?
  \item Inter- or intra-IOC record relationships
\end{itemize}

Additionally, whatrecord offers tools for:
\begin{itemize}
  \item EPICS build system Makefile introspection, a sumo\cite{sumo}-inspired
    implementation.
  \item GDB Python script that inspects binary symbols to find IOC shell commands, variables and source code context
    \begin{verbatim}dbLoadRecords [str: fname] [str: subs]\end{verbatim}
    \begin{verbatim}.../src/ioc/db/dbIocRegister.c:53\end{verbatim}
  \item Accurate EPICS macro handling (epics-base macLib, wrapped with Cython in epicsmacrolib\cite{epicsmacrolib}).
  \item Linting startup scripts.
  \item Plugins for loading happi devices, TwinCAT PLC projects, IOC information from LCLS’s IOC manager, …
  \item Process database record to Beckhoff TwinCAT PLC source code definition (when used in conjunction with pytmc\cite{pytmc})
  \item Python API, command-line tools for some of the above things
  \item A web-based API/backend server to monitor IOC scripts and serve IOC/record information.
  \item Vue.js-based frontend single-page application
\end{itemize}


Parsing with jq

\begin{lstlisting}[language=bash]
  $ whatrecord parse whatrecord/tests/iocs/db/pva/iq.db |
     jq '.records[] | [.name, .record_type, .fields.OUT.value]'
  [
    "$(PREFIX)Rate",
    "ao",
    "$(PREFIX)dly_.ODLY NPP"
  ]
  [
    "$(PREFIX)Delta",
    "ao",
    null
  ]
  ...
\end{lstlisting}

\begin{lstlisting}[language=bash]
$ whatrecord parse whatrecord/tests/iocs/db/pva/iq.db | 
    jq '.records[] | [ .name, .info["Q:group"]]'
[
  "$(PREFIX)Rate",
  null
]
[
  "$(PREFIX)Phase:I",
  {
    "$(PREFIX)iq": {
      "phas.i": {
        "+type": "plain",
        "+channel": "VAL"
      }
    }
  }
]
...

\end{lstlisting}

\section{CORE FUNCTIONALITY}

\subsection{Parsing with lark}

\subsection{Backend serverParsing with lark}

\begin{itemize}
  \item Load up all EPICS IOCs (either user-specified or those listed in LCLS’s IOC manager tool)
  \item Load the startup scripts
  \item Load all the databases and supported files
  \item Monitor loaded files for changes
  \item Provide a backend service for querying the information
  \item Based on the backend server, provide a frontend for easy access to that information
\end{itemize}

\subsection{Command-line tools}
\subsubsection{whatrecord deps}

Makefile-derived dependency graph tool

\subsubsection{whatrecord graph}

Intra/inter-IOC record graphs

State notation language transition diagrams

\subsubsection{whatrecord server}

\section{Vue.js WEB FRONTEND}

\begin{itemize}
  \item Search for records/IOCs/etc by name and dig into the details...
\end{itemize}

\subsection{Record searching}
\subsection{IOC listing}
\subsection{Inter-IOC PV Map}
\subsection{Gateway}
\subsection{LCLS-Specific Tools}
\subsubsection{Happi}
\subsubsection{LDAP / netconfig settings viewer}
\subsubsection{epicsArch settings}

\subsection{References}

The whatrecord source code is available on GitHub\cite{whatrecord-github}
and documentation is available on GitHub Pages\cite{whatrecord-docs}.

%
% only for "biblatex"
%
\ifboolexpr{bool{jacowbiblatex}}%
	{\printbibliography}%
	{%
	% "biblatex" is not used, go the "manual" way
	
	%\begin{thebibliography}{99}   % Use for  10-99  references
	\begin{thebibliography}{9} % Use for 1-9 references
	
	\bibitem{whatrecord-github}
		whatrecord source code repository,
		\url{http://www.github.com/pcdshub/whatrecord}
	
	\bibitem{whatrecord-docs}
		whatrecord documentation,
		\url{http://pcdshub.github.io/whatrecord}
	
	\bibitem{epics-base}
		EPICS,
		\url{http://www.aps.anl.gov/epics/}
	
	\bibitem{streamdevice}
		StreamDevice,
		\url{https://paulscherrerinstitute.github.io/StreamDevice/index.html}
	
	\bibitem{lark}
		Lark - a parsing toolkit for Python,
		\url{https://github.com/lark-parser/lark/}
	
	\bibitem{epicsmacrolib}
		epicsmacrolib,
		\url{https://github.com/pcdshub/epicsmacrolib}
	
	\bibitem{sumo}
		epics-sumo,
		\url{https://epics-sumo.sourceforge.io/}
	
	\bibitem{pytmcsumo}
		pytmc,
		\url{https://github.com/pcdshub/pytmc/}
	
	\end{thebibliography}
} % end \ifboolexpr

%
% for use as JACoW template the inclusion of the ANNEX parts have been commented out
% to generate the complete documentation please remove the "%" of the next two commands
% 
%%%%\newpage

%%%%\include{annexes-Letter}

\end{document}
