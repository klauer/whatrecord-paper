% !TeX spellcheck = en_US
%
%
% v 2.3  Feb 2019   Volker RW Schaa
%		# changes in the collaboration therefore updated file "jacow-collaboration.tex"
%		# all References with DOIs have their period/full stop before the DOI (after pp. or year)
%		# in the author/affiliation block all ZIP codes in square brackets removed as it was not 
%         understood as optional parameter and ZIP codes had bin put in brackets
%       # References to the current IPAC are changed to "IPAC'19, Melbourne, Australia"
%       # font for "url" style changed to "newtxtt" as it is easier to distinguish "O" and "0"
%
\documentclass[letter,
               %boxit,        % check whether paper is inside correct margins
               %titlepage,    % separate title page
               %refpage       % separate references
               %biblatex,     % biblatex is used
               keeplastbox,   % flushend option: not to un-indent last line in References
               %nospread,     % flushend option: do not fill with whitespace to balance columns
               %hyphens,      % allow \url to hyphenate at "-" (hyphens)
               %xetex,        % use XeLaTeX to process the file
               %luatex,       % use LuaLaTeX to process the file
               ]{jacow}
%
% ONLY FOR \footnote in table/tabular
%
\usepackage{pdfpages,multirow,ragged2e} %
%
% CHANGE SEQUENCE OF GRAPHICS EXTENSION TO BE EMBEDDED
% ----------------------------------------------------
% test for XeTeX where the sequence is by default eps-> pdf, jpg, png, pdf, ...
%    and the JACoW template provides JACpic2v3.eps and JACpic2v3.jpg which
%    might generates errors, therefore PNG and JPG first
%
\makeatletter%
	\ifboolexpr{bool{xetex}}
	 {\renewcommand{\Gin@extensions}{.pdf,%
	                    .png,.jpg,.bmp,.pict,.tif,.psd,.mac,.sga,.tga,.gif,%
	                    .eps,.ps,%
	                    }}{}
\makeatother

% CHECK FOR XeTeX/LuaTeX BEFORE DEFINING AN INPUT ENCODING
% --------------------------------------------------------
%   utf8  is default for XeTeX/LuaTeX
%   utf8  in LaTeX only realises a small portion of codes
%
\ifboolexpr{bool{xetex} or bool{luatex}} % test for XeTeX/LuaTeX
 {}                                      % input encoding is utf8 by default
 {\usepackage[utf8]{inputenc}}           % switch to utf8

\usepackage[USenglish]{babel}

% My packages:
\usepackage{listings}
\usepackage{color}

\definecolor{dkgreen}{rgb}{0,0.6,0}
\definecolor{gray}{rgb}{0.5,0.5,0.5}
\definecolor{mauve}{rgb}{0.58,0,0.82}

\lstset{frame=tb,
  language=Python,
  aboveskip=3mm,
  belowskip=3mm,
  showstringspaces=false,
  columns=flexible,
  basicstyle={\small\ttfamily},
  numbers=none,
  numberstyle=\tiny\color{gray},
  keywordstyle=\color{blue},
  commentstyle=\color{mauve},
  stringstyle=\color{dkgreen},
  breaklines=true,
  breakatwhitespace=true,
  tabsize=3
}

%
% if BibLaTeX is used
%
\ifboolexpr{bool{jacowbiblatex}}%
 {%
  \addbibresource{jacow-test.bib}
  \addbibresource{biblatex-examples.bib}
 }{}
\listfiles

%%
%%   Lengths for the spaces in the title
%%   \setlength\titleblockstartskip{..}  %before title, default 3pt
%%   \setlength\titleblockmiddleskip{..} %between title + author, default 1em
%%   \setlength\titleblockendskip{..}    %afterauthor, default 1em

\begin{document}

\title{WHATRECORD: A PYTHON-BASED EPICS FILE FORMAT TOOL
\thanks{Work supported by U.S. D.O.E. Contract DE-AC02-76SF00515.}}
\author{Kenneth Lauer\thanks{klauer@slac.stanford.edu}, SLAC National Accelerator Laboratory, Menlo Park, CA }
	
\maketitle

%
\begin{abstract}
  \verb_whatrecord_\cite{whatrecord-github} is a Python-based parsing tool for
  interacting with a variety of EPICS (Experimental Physics and Industrial
  Control System) file formats, including V3 and V7 database files. The project
  aims for compliance with epics-base by using Lark\cite{lark} grammars that
  closely reflect the original Lex/Yacc grammars. 

  \verb_whatrecord_ offers a suite of tools for working with its supported file
  formats, with convenient Python-facing dataclass object representations and
  easy JSON (JavaScript Object Notation) serialization. A prototype backend web
  server for hosting IOC (Input/Output Controller) and record information is
  also included as well as a Vue.js-based frontend, an EPICS build system
  \verb_Makefile_ dependency inspector, a static analyzer-of-sorts for startup
  scripts, and a host of other things that the author added at whim to this
  side project.
\end{abstract}

\section{BACKGROUND}

\subsection{The problem - and the inspiration}

At the LCLS (SLAC's Linac Coherent Light Source), the accelerator and photon
side control systems include approximately 3000 IOC instances in total, with
hundreds of modules and dozens of versions per module.

In general, these EPICS IOCs, modules, and extensions are comprised of a
conglomeration of unique file formats. Some common examples of such file
formats include:

\begin{Itemize}
  \item Process database files (\verb_.db_)
  \item Database definition files (\verb_.dbd_)
  \item Template / substitutions files
  \item IOC shell scripts (\verb_st.cmd_)
  \item StreamDevice protocols (\verb_.proto_)
  \item State notation language programs (\verb_.st_)
  \item Gateway configuration (\verb_.pvlist_)
  \item Access security files (\verb_.acf_)
  \item Build system \verb_Makefile_s
\end{Itemize}

Additionally, facility-specific tools (centralized IOC management tools like
LCLS's IOC Manager, archiver appliance automation tools, and so on) build on
top of IOCs and records.

Combined, this makes for an enormous code base with a mix of these
EPICS-specific file formats.

Links between these files are often implicit.  Take, for example, that an EPICS
IOC record has a specific record type name alongside its name in a database
file (\verb_.db_), an EPICS PV (Process Variable) name, in a traditional IOC, starts
with the record name defined in a database file. This PV name acts as a global
identifier that allows for clients on the same network subnet to access - and
potentially modify - related data.

A record is made up of fields which can contain metadata like engineering units
or user-specified descriptions, references to other records, relevant data
values, and so on.

An example record instance, defining a single AI (analog input) record named
\verb_IOC:RECORD:NAME_ is as follows:
\begin{lstlisting}[language=bash]
  record(ai, "IOC:RECORD:NAME") {}
\end{lstlisting}

This file does not define what the fields of the record type; that is the
responsibility of the database definition file (.dbd). A simplified excerpt
from a database definition file, defining a single field
for the "ai" record type is as follows:

\begin{lstlisting}[language=bash]
  recordtype(ai) {
      ...
      field(NAME, DBF_STRING) {
          special(SPC_NOMOD)
          size(61)
          prompt("Record Name")
      }
      ...
  }
\end{lstlisting}

Note that there is no explicit link between the database file and the database
definition file: neither reference the other by filename.
Rather, one can only infer the link by examining a third file, the IOC-specific
IOC shell script (.cmd) file, line-by-line.

An excerpt from such a startup script could look like:
\begin{lstlisting}[language=bash]
dbLoadDatabase("path/to/the.dbd",0,0)
IOC_registerRecordDeviceDriver(pdbbase) 
dbLoadRecords("records.db")
\end{lstlisting}

Each line of this script includes up to one command. Each of those commands
has been registered by either EPICS itself, the modules included in the IOC,
or the IOC source code itself.  Typically, the available commands
would be found either in documentation or by executing the IOC and invoking the
built-in help system. Alternatively, the most reliable fallback ends up
being the source code itself.

Other direct or indirect references may be found inside fields.  
For example, depending on the \verb_DTYP_ (device type) field, the \verb_INP_
(input specification) field may be a custom string defined at the device
support layer. Interpretation of this field requires knowledge of how
these are formatted.  Take StreamDevice\cite{streamdevice}, a generic
support module for communicating with controllers that use simple byte
streams for communication, for example:
\begin{lstlisting}[language=bash]
  record(ai, "IOC:RECORD:NAME") {
    field(DTYP, "stream")
    field(INP,  "@ProtocolFilename.proto getValue PS1")
  }
\end{lstlisting}
The device type here is set to \verb_"stream"_, a custom identifier that
StreamDevice has hard-coded.  This instructs EPICS to use StreamDevice
and interpret the \verb_INP_ field with it.  It us up to the IOC developer to
understand the format of these strings and set them appropriately, in order
to reference back to the protocol file that defines the byte string to send
and the expected response format. Here, a StreamDevice protocol file for
the above record indicates that a simple string \verb_WHAT:IS:THE:VALUE?_
is to be sent, and a floating point value (\verb_%f_, as in the C \verb_scanf_
format specifiers) is to be sent from the controller:
\begin{lstlisting}[language=bash]
  getValue {
      out "WHAT:IS:THE:VALUE?";
      in "%f";
  }
\end{lstlisting}

This section is a small but important part of what makes up an IOC: the build
system surrounding all of these files, other modules with their own standards,
access security configuration for intra-subnet access, gateway configuration
controlling inter-subnet access, facility-specific tools that rely on PV
names, and so on further complicate the number of files and references
one needs to be aware of.

While those familiar with EPICS IOC development may find that the above is
obvious and simple, it can be opaque at best to those unable to dedicate the
time to reading through esoteric (and often outdated) manuals or source
code.

\subsection{Goals and the emergence of whatrecord}

The previous section's problem led the author over the years to desire a tool
that could somehow unify these file formats and provide the ability to inspect
the links.

These initial goals led to the creation of this new Python package,
\verb_whatrecord_:
\begin{itemize}
  \item Allow for easy parsing of all the special formats outside, and
    represent them in a widely-used interchange format like JSON.
  \item Aid the user in the understanding of existing IOCs, whether they are
    deployed and running or not.
  \item Provide a method to see how different records, different IOCs, all
    relate to one another, without requiring the IOC to be running.
  \item Provide a method for cross-referencing a PV name to its database
    file, record definition, startup script, and IOC.
\end{itemize}

With these implemented, pathways for new possibilities were opened: the ability
to linking records to PLC code, to StreamDevice protocol information, to
gateway access rules, and even shell commands to their respective source code.

\section{CORE FUNCTIONALITY}

\subsection{Overview}
\verb_whatrecord_ will parse any of the following into intuitive Python dataclasses
using the Lark\cite{lark} parsing toolkit:
\begin{itemize}
  \item Database files (V3 or V4/V7), database definitions,
    template/substitution files
  \item Access security configuration files
  \item Autosave .sav files
  \item Gateway pvlist configuration files
  \item StreamDevice protocol files
  \item snlseq/sequencer state machine parsing
\end{itemize}

IOC shell scripts (i.e., \verb_st.cmd_) can be interpreted and annotated with
contextual information during the loading process.  The process aims to record
what files were loaded during startup, what records will be available in the
IOC, what errors were found when loading, what file and line did each record
get loaded, and what are the inter- or intra-IOC record relationships.

Additionally, \verb_whatrecord_ offers tools for:
\begin{itemize}
  \item Exporting all parsed results to JSON-serializable objects.
  \item EPICS build system \verb_Makefile_ introspection, a sumo\cite{sumo}-inspired
    implementation.
  \item GDB Python script that inspects binary symbols to find IOC shell
    commands, variables and source code context
    \begin{lstlisting}[language=bash]
  dbLoadRecords [str: fname] [str: subs]
  .../src/ioc/db/dbIocRegister.c:53
    \end{lstlisting}
  \item Accurate EPICS macro handling using epicsmacrolib\cite{epicsmacrolib}.
  \item Linting startup scripts.
  \item Plugins for loading happi devices, TwinCAT PLC projects, and IOC
    information from LCLS’s IOC manager.
  \item Process database record to Beckhoff TwinCAT PLC source code definition
    mapping (when used in conjunction with pytmc\cite{pytmc}).
\end{itemize}

An intuitive Python API, user-facing command-line tools, a web-based
API/backend server to monitor IOC scripts and serve IOC/record information, and
a Vue.js-based frontend single-page application are also provided.

\subsection{Parsing with lark}

\verb_whatrecord_ utilizes the Lark\cite{lark} parser internally to parse most of its supported
file formats.  Lark supports writing custom parsers with EBNF (extended
Backus–Naur form) metasyntax.  It is capable of parsing all context-free
grammars with ambiguity resolution. 

The grammars implemented in \verb_whatrecord_ closely resemble those in EPICS because
the source Lex/Yacc grammars are syntactically similar.  A test suite is
included in \verb_whatrecord_ which attempts to cover various aspects of the packaged
grammars.

Parsing a file in \verb_whatrecord_ results in user-friendly type annotated
\verb_dataclass_ instances. These can be readily inspected programmatically,
serialized to JSON, and - in several cases - exported back into their original
format.

\begin{lstlisting}[language=python]
import whatrecord
db = whatrecord.parse(
    "whatrecord/tests/iocs/db/basic_asyn_motor.db"
)
record = db.records["$(P)$(M)"]
print(record.fields["TWV"].value)  # -> '1'
\end{lstlisting}

A key feature of \verb_whatrecord_'s parsing utilities is that it records the context
of many operations.  When two different files are used to load a record,
both will be present in the context information:

\begin{lstlisting}[language=python]
import whatrecord
ioc = whatrecord.parse(
  "whatrecord/tests/iocs/ioc_a/st.cmd"
)
db = ioc.shell_state.database
record = db["IOC:KFE:A:One"]
print(record.context)
\end{lstlisting}

Yields the following, indicating the files and line numbers used to load the
given database record instance:
\begin{lstlisting}[language=python]
(whatrecord/tests/iocs/ioc_a/st.cmd:13,
 whatrecord/tests/iocs/ioc_a/ioc_a.db:1)
\end{lstlisting}

\subsection{Exporting/serializing to JSON}

With the \verb_whatrecord_ command-line entry point, you can parse supported formats
and pipe their information to tools like \verb_jq_\cite{jq} to interact and run
basic queries on data.

The following lists all records in a database file and selects a set of
information:
\begin{lstlisting}[language=bash]
  $ whatrecord parse whatrecord/tests/iocs/db/pva/iq.db |
     jq '.records[] | [.name, .record_type, .fields.OUT.value]'
  [
    "$(PREFIX)Rate",
    "ao",
    "$(PREFIX)dly_.ODLY NPP"
  ]
  [
    "$(PREFIX)Delta",
    "ao",
    null
  ]
  ...
\end{lstlisting}

The following inspects some EPICS V4 \verb_Q:Group_ settings:

\begin{lstlisting}[language=bash]
$ whatrecord parse whatrecord/tests/iocs/db/pva/iq.db | 
    jq '.records[] | [ .name, .info["Q:group"]]'
[
  "$(PREFIX)Rate",
  null
]
[
  "$(PREFIX)Phase:I",
  {
    "$(PREFIX)iq": {
      "phas.i": {
        "+type": "plain",
        "+channel": "VAL"
      }
    }
  }
]
...

\end{lstlisting}

\section{FORMATS AND IMPLEMENTATION NOTES}

\subsubsection{Database files}

\verb_whatrecord_ implements two grammars for EPICS process databases, as there are
significant differences between the grammars used for EPICS V3 and V4+ IOCs.

A flag is available \verb_--v3_ for \verb_whatrecord parse_ to aid facilities
that have not yet opted in to using V4+.

\subsubsection{Access Security Configuration Files (ACF)}

whatrecord parses ACF files that target typical EPICS IOCs along with
the EPICS gateway.  This information can be correlated to records in the
frontend, covered in a later section.

\subsubsection{Substitution files}

Substitutions files and those supported by \verb_dbLoadTemplate_ are
readily supported by \verb_whatrecord_.  Contextual information as to which
files are loaded in the process is reliably carried along.

A separate grammar for the format used by the command-line tool \verb_MSI_ (the
Macro Substitution and Include tool) is also provided due to their
differing implementations.

\subsubsection{Autosave save files}

Autosave save files (\verb_.sav_), which track the state of a record such that
it may be restored at the next boot of an IOC, are supported by \verb_whatrecord_.
In the web frontend, these values will be displayed alongside those in the
database file.

\subsubsection{Gateway PV lists}

Gateway \verb_.pvlist_ files are supported by \verb_whatrecord_.  In the frontend,
this allows for the easy correlation of gateway rules to records that
apply to it, and also in reverse, showing which gateway rules apply to
the selected record.

\subsubsection{Sequencer - State Notation Language}

The EPICS sequencer's state notation language format \verb_.st_ files is
supported by \verb_whatrecord_ with some caveats.  \verb_whatrecord_ does not have full
support for a C preprocessor, which the sequencer relies on.  This means
that for simple files (and those that only use simple C \verb_define_s), \verb_whatrecord_
successfully parses the files, whereas complicated \verb_define_s could fail
to parse with the included grammar.

\subsubsection{LCLS-specific formats}

At the LCLS, there are additional file formats that are in common use.
The \verb_epicsArch_ format enumerates process variables that are to be
included for recording in its data acquisition system.  \verb_whatrecord_ provides
support for this format and allows for linting and display through the
frontend.

The LCLS photon controls team uses the Python suite Bluesky\cite{bluesky} for
slow-speed data acquisition, with approximately 1,000 Ophyd devices indexed in
a happi\cite{happi} database.  \verb_whatrecord_ provides happi integration, mapping
process variables used in Ophyd devices back to their IOCs.

\subsubsection{Startup scripts}

Startup scripts do not have a corresponding grammar. In the EPICS
implementation, the IOC shell parses commands character-by-character with a
custom routine.  \verb_whatrecord_ has a ported version of this parsing function
for maximum compatibility, supporting redirection and everything else from the
original.

\subsubsection{Macro handling}

Files supported by \verb_whatrecord_ often use the EPICS macro library (macLib)
in order to interpolate variables (environment variables or otherwise) to their
corresponding values.

\verb_whatrecord_ uses epicsmacrolib\cite{epicsmacrolib}, which is a Cython-wrapped
version of EPICS macLib, under the hood to dutifully reproduce standard EPICS
macro expansion and state tracking.

\subsubsection{Makefiles}

\verb_Makefile_s can be inspected to reveal per-module or per-IOC dependencies
and settings of the EPICS build system. Unlike the other supported formats, 
\verb_Makefile_s are not parsed but rather introspected by way of GNU
\verb_make_, an optional external requirement.

The implementation is noted as sumo\cite{sumo}-inspired, as an important part
of how sumo scans source code is replicated. The library executes \verb_make_
with a custom target, enabling the direct handling of any and all
\verb_Makefile_ syntax that the host supports.

\verb_whatrecord_ is able to extract a variety of information from a \verb_Makefile_
in this manner, exporting information such as the EPICS build architectures,
cross-compiler host architectures, target architectures, the EPICS base
version, configuration paths, release top variables, environment variable
settings, and so on.

\subsection{Backend server}

Building on top of the parsing tools, \verb_whatrecord_ provides an
aiohttp\cite{aiohttp}-based "backend" server daemon that enables users to query
IOC-related information over a RESTful (REpresentational State Transfer) JSON
interface.

A summary as to how the backend server operates is as follows:

\begin{enumerate}
  \item Find all EPICS IOCs specified by the user (or those listed in LCLS's
    IOC manager tool).
  \item Load the startup scripts with the built-in parsing tools, including
    databases and supported files.
  \item Periodically check previously-loaded files for changes, and re-load
    IOCs.
  \item Listen for clients (or the \verb_whatrecord_ frontend) querying for IOC
    information by way of aiohttp.
\end{enumerate}

\subsection{Command-line tools}
\subsubsection{whatrecord deps}

This tool utilizes the \verb_Makefile_ parsing tools internally to recursively
generate a dependency graph of supporting modules to a given IOC.

\subsubsection{whatrecord graph}

This graphing tool allows for graphing of records or state notation language
transition diagrams. Inter-IOC record links may be graphed if multiple
database files or startup scripts are specified.

A sample state notation language graph is shown in Fig. \ref{fig:simple.st},
and a sample record relationship graph is shown in Fig. \ref{fig:records}.

\begin{figure}
   \centering
   \includegraphics*[width=.7\columnwidth]{st-simple}
   \caption{A sample state notation language program graph.}
   \label{fig:simple.st}
\end{figure}

\begin{figure}
   \centering
   \includegraphics*[width=.9\columnwidth]{calc-records}
   \caption{A sample record relationship graph.}
   \label{fig:records}
\end{figure}

\subsubsection{whatrecord server}

This command spawns the backend server.  It may also be used to export a cached
state for offline usage by the frontend.

\subsubsection{whatrecord lint}

This command offers a work-in-progress set of linting tools, most of which
has not yet been well-defined.  The goal for these tools will be to
allow a facility to enforce certain standards for their IOC files
and avoid common pitfalls by detecting issues prior to the IOC boot process.

\section{WEB FRONTEND}

A Vue.js\cite{vuejs} web-based frontend application is packaged separately in
the \verb_whatrecord_ repository.  It provides a user-friendly view of the information
that \verb_whatrecord_ can parse and aggregate.

A searchable index of records is the primary view.  Record information can be
used to correlate back to other tools and views. For example, when an autosave
configuration is found in an IOC, the frontend will display that information in
a table underneath "Autosave" and annotate the record to show the on-restore
value.

Similarly, StreamDevice protocol files, happi devices, asyn and motor port,
Archiver Appliance, access security groups, and others will be displayed
alongside the record information.  Source code files may be viewed by
clicking on any context link.

Record links, when detected, are displayed as an interactive graph (by way of
\verb_d3-graphviz_\cite{d3graphviz}).  Inter-IOC links can be interactively
investigated in the "PV Map" section.

Sample screenshots of the frontend are shown in Figs. \ref{fig:whatrecord-asg}
and \ref{fig:whatrecord-links}.

\begin{figure}
   \centering
   \includegraphics*[width=.9\columnwidth]{whatrecord-asg}
   \caption{whatrecord frontend: a record with access security group settings.}
   \label{fig:whatrecord-asg}
\end{figure}

\begin{figure}
   \centering
   \includegraphics*[width=.9\columnwidth]{whatrecord-ioc-stream-log0}
   \caption{whatrecord frontend: a record with links.}
   \label{fig:whatrecord-links}
\end{figure}

\subsection{LCLS}
Views for LCLS-Specific tools can be enabled with an environment variable
setting, including a view of happi devices, an LDAP / netconfig settings
viewer, and epicsArch PV listings.

\subsection{GitHub Actions and "offline" mode}

A sample repository\cite{gha-sample} is provided to exhibit the "offline" mode
supported by \verb_whatrecord_.

The offline mode utilizes the \verb_whatrecord_ server in continuous integration to
generate a snapshot of the server information. This information then takes
the place of a backend server, requiring only a tarball of JSON in order to
populate the data.

\subsection{Trying whatrecord}

The easiest method to try the frontend alongside the backend is with Docker
Compose.

\begin{lstlisting}[language=bash]
$ git clone https://github.com/pcdshub/whatrecord
$ cd whatrecord/docker
$ docker-compose up
\end{lstlisting}
After executing the above, wait for a few minutes and then open
\verb_http://localhost:8896_ in a browser.

The parsing tools can be used directly with a working Python 3.9+ environment:

\begin{lstlisting}[language=bash]
	$ pip install whatrecord
	$ whatrecord --help
\end{lstlisting}

The \verb_whatrecord_ source code is available on GitHub\cite{whatrecord-github}
and documentation is available on GitHub Pages\cite{whatrecord-docs}.

\ifboolexpr{bool{jacowbiblatex}}%
	{\printbibliography}%
	{%
	% "biblatex" is not used, go the "manual" way
	
	%\begin{thebibliography}{99}   % Use for  10-99  references
	\begin{thebibliography}{9} % Use for 1-9 references
	
	\bibitem{whatrecord-github}
    whatrecord: source code repository,
		\url{http://www.github.com/pcdshub/whatrecord}
	
	\bibitem{whatrecord-docs}
    whatrecord: documentation,
		\url{http://pcdshub.github.io/whatrecord}
	
	\bibitem{epics-base}
		EPICS,
		\url{http://www.aps.anl.gov/epics/}
	
	\bibitem{streamdevice}
		StreamDevice support,
		\url{https://paulscherrerinstitute.github.io/StreamDevice/index.html}
	
	\bibitem{lark}
		Lark: a parsing toolkit for Python,
		\url{https://github.com/lark-parser/lark/}
	
	\bibitem{epicsmacrolib}
    epicsmacrolib: EPICS-compliant macro handling,
		\url{https://github.com/pcdshub/epicsmacrolib}
	
	\bibitem{sumo}
		sumo: EPICS support module manager,
		\url{https://epics-sumo.sourceforge.io/}
	
	\bibitem{aiohttp}
		aiohttp: an asyncio HTTP client/server,
		\url{https://docs.aiohttp.org/en/stable/}
	
	\bibitem{pytmc}
    pytmc: TwinCAT PLC Code to EPICS Database Tool,
		\url{https://github.com/pcdshub/pytmc/}
	
	\bibitem{gha-sample}
		GitHub Actions and whatrecord sample,
		\url{https://github.com/pcdshub/ioc-whatrecord-example}
	
	\bibitem{jq}
		jq: JSON filter command-line tool,
		\url{https://jqlang.github.io/jq/manual/}
	
	\bibitem{bluesky}
    Bluesky,
		\url{https://blueskyproject.io/}
	
	\bibitem{vuejs}
    Vue.js,
		\url{https://vuejs.org/}
	
	\bibitem{d3graphviz}
    d3-graphviz,
		\url{https://github.com/magjac/d3-graphviz}
	
	\end{thebibliography}
} % end \ifboolexpr

%
% for use as JACoW template the inclusion of the ANNEX parts have been commented out
% to generate the complete documentation please remove the "%" of the next two commands
% 
%%%%\newpage

%%%%\include{annexes-Letter}

\end{document}
